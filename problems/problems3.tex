\documentclass[12pt]{article}
\usepackage{enumitem}
\usepackage{amsmath,amsgen,amstext,amsbsy,amsopn,amsfonts}  
\usepackage{url,graphicx,tabularx,array,geometry}  
\renewcommand{\familydefault}{\sfdefault}

\newcommand{\Z}{\mathbb{Z}}

\pagestyle{plain}

\setlength{\parskip}{1ex} %--skip lines between paragraphs
\setlength{\parindent}{25px} %--don't indent paragraphs

\title{Problem set 3}
\date{}
\author{Brandon E. Feder}

\begin{document}
    \maketitle
    
    \section*{Exploration}
    \begin{enumerate}
        \item Question 1
        \item Question 2
    \end{enumerate}

    \section*{Numerical}

    \begin{enumerate}[resume]
        \item \Find an integral solution to the equation $7469x+2463y=1$. Is there more than one answer? Find the multiplicative inverse of $2463\;(\;\text{mod}\;7469\;)$.
        
        Find one such value of $x$, and $y$ can be done using the extended euclidian algorithm. One such solution is $x=1016$, and $y=-3081$. That is, $7469(1016)+2463(-3081)=1$. There are, however, many such values of $x$, and $y$. This solution is similar to finding the multiplicative invere of $2463\;(\;\mathrm{mod}\;7469\;)$:

        
        $\mathrm{define}\;x$ such that $2463\equiv \;x^{-1}(\;\mathrm{mod}\;7469\;)$.

        % A solution is (1016, -3081). There are many solutions. Multiplicative inverse is 6162.
        
        
        
        
        \item 

    \end{enumerate}

    
\end{document}
